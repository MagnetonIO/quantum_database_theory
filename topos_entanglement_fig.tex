\documentclass[tikz]{standalone}
\usepackage{tikz}
\usetikzlibrary{shapes.geometric, backgrounds}

\begin{document}
\begin{tikzpicture}[font=\small, node distance=2cm]

% Styles
\tikzset{
  context/.style={
    ellipse,
    draw,
    thick,
    minimum width=2.8cm,
    minimum height=2cm,
    align=center
  },
  entline/.style={
    thick,
    decorate,
    decoration={snake, amplitude=1.5pt}
  }
}

% Context circles (for local presheaves or measurement contexts)
\node[context] (c1) {Context\\$C_1$};
\node[context, right=2cm of c1] (c2) {Context\\$C_2$};
\node[context, below right=1cm and -0.5cm of c1] (c3) {Context\\$C_3$};

% Overlapping is done so that c2 partially overlaps with c1, and c3 is below them.

% Add "entanglement lines" between contexts
\draw[entline] (c1) -- (c2) node[midway, above] {Entanglement};
\draw[entline] (c1) -- (c3) node[midway, left] {Entanglement};
\draw[entline] (c2) -- (c3) node[midway, right] {Entanglement};

% A dashed circle around them for "Global Section"
\node[draw, dashed, circle, fit=(c1)(c2)(c3), label=above:{Global Section}, inner sep=0.6cm] {};

\end{tikzpicture}
\end{document}
